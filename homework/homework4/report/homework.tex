\documentclass[10pt,a4paper]{article}
\usepackage[latin1]{inputenc}
\usepackage{amsmath}
\usepackage{amsfonts}
\usepackage{amssymb}
\usepackage{graphicx}
\author{Wu Bingzhe\\1200010666\\The School of mathmatical science}
\title{Pattern Recognition homework\\Chapter 4}
\begin{document}
	\maketitle
	\section{Ex1}
	
	According to $x_j\in R_i,j=1,2$,we have :
	\begin{equation}
	g_i(x_j)\leq(\geq)0
	\end{equation}
	In terms of $g(\lambda x_1+(1-\lambda)x_2)=\lambda g(x_1)+(1-\lambda)g(x_2)$, combine (1) and $0\leq\lambda\leq 1$,we get:
	\begin{equation}
	g(\lambda x_1+(1-\lambda)x_2)\leq(\geq)0
	\end{equation}	
	So the decision area is convex 
	\section{Ex2}
	\subsection{(a)}
	The projection of $x_a$ can be represented as :
	\begin{equation}
	x_a=x_p+r\dfrac{\omega}{\|\omega\|}
	\end{equation}
	Where $x_p$ is the projection point on the hyperplane ,$r$ is the distance between the $x_a$ and
	the hyperplane $g(x)=0$.
	
	Because the point $x_p$ is on the hyperplane ,we have $g(x_p)=0$.Further more:
	\begin{equation*}
	\begin{split}
		g(x_a)&=\omega^Tx_a+\omega_0\\
		&=\omega^T(x_p+r\dfrac{\omega}{\|\omega\|})+\omega_0\\
		&=g(x_p)+r\|\omega\|\\
		&=r\dfrac{\omega}{\|\omega\|}
	\end{split}
	\end{equation*}
	So we get $r=\dfrac{|g(x_a)|}{\|\omega\|}$
	
	On the other hand,for an arbitrarily point $x_q$ on the hyperplane ,
	\begin{equation*}
	\begin{split}
	\|x_q-x_a\|^2&=\|x_q-x_p+x_p-x_a\|^2\\
	&=\|x_q-x_a\|^2+r^2+(x_q-x_p)(x_p-x_a)\\
	&=\|x_q-x_a\|^2+r^2\\
	&\geq r^2
	\end{split}
	\end{equation*}
	The conclusion established.
	\subsection{(b)}
	In terms of $r=\dfrac{g(x_a)}{\|\omega\|}$ , combine (3) ,we can get the projection of $x_a$ is :
	\begin{equation*}
	x_p=x_a-\dfrac{g(x_a)}{\|\omega\|^2}\omega
	\end{equation*} 
\section{Ex3}
\subsection{(a)}
Assumption $y_i$ satisfy $a^Ty_i\geq b$,hence,we have $a^Ty_i\geq b\geq 0$.
So after bringing the $b$,the solution area is located in the original problem area.
\subsection{(b)}
The boundary of the solution after bringing the $b$ is a  hyperplane $a^Ty_i=b$,and the original boundary is another hyperplane $a^Ty_i=0$.
So the distance between the two hyperplane is $d=\dfrac{b}{\|y_i\|}$
\section{Ex4}

The criterion function of perceptron has the form :
\begin{equation*}
J(a)=\sum_{y\in Y}(-a^ty)
\end{equation*}
\section{Ex5}
\section{Ex6}
\end{document}