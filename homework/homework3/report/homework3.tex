\documentclass[10pt,a4paper]{article}
\usepackage[latin1]{inputenc}
\usepackage{amsmath}
\usepackage{amsfonts}
\usepackage{amssymb}
\usepackage{graphicx}
\author{Wu Bingzhe\\1200010666\\The school of mathmatical science}
\title{Pattern Recognition Report3\\Chapter3}
\begin{document}
	\maketitle
	\section{Question1}
    In my program of this question,function u1.m compute question(a),u2.m for question(b)
    ,u3.m for question(c),u4.m for question(d).
    \subsection{(a)}
		According to the maximum likelihood estimation on the Gaussian distribution,we can 
		get :
		\begin{equation}
			\hat{\mu}=\dfrac{1}{N}\sigma_{k=1}{N}x_k
		\end{equation}	
		\begin{equation}
			\hat{\Sigma}=\dfrac{1}{N}\Sigma_{k=1}^{N}(x_k-\hat{\mu})(x_k-\hat{\mu})^{T}
		\end{equation}
		In terms of (1) and (2),we could compute the value of the parameters 
		by the program . The results are as follows:
	    \begin{center}
		\begin{tabular}{|c|c|c|}
			\hline  feature& $\hat{\mu}$  & $\hat{\sigma}^2$  \\ 
			\hline  $x_1$& -0.0709 & 0.9062   \\ 
			\hline  $x_2$&  -0.6047& 4.2007  \\ 
			\hline  $x_3$&  -0.9110&  4.5419 \\ 
			\hline 
		\end{tabular} 
		\end{center}	
	\subsection{(b)}
	Similar with (1) and (2),we can get the results as follows:
	
	$\mu_{12}=(-0.0709,-0.6047)^T$ \ $\Sigma_{12}=$
	$\begin{array}{cc}
		0.9062& 0.5678  \\ 
	    0.5678& 4.2007 
	\end{array} 
	$
	
	$\mu_{23}=(-0.6047,-0.9110)^T$ \ $\Sigma_{23}=$
	$\begin{array}{cc}
	4.2007& 0.7337  \\ 
	0.7337& 44.5419 
	\end{array} 
	$
	
	$\mu_{13}=(-0.0709,-0.9110)^T$ \ $\Sigma_{13}=$
	$\begin{array}{cc}
	0.9062& 0.3941  \\ 
	0.3941& 4.5419 
	\end{array} 
	$
	\subsection{(c)}
	According to (1) and (2),we get the results as follows:
	\begin{equation*}
	\mu=(-0.0709,-0.6047,-0.9110)^T
	\end{equation*}
	\begin{equation*}
	\Sigma=\begin{array}{ccc}
	0.9062& 0.5678  & -0.9110  \\ 
	0.5678& 4.2007 & 0.7337  \\ 
	0.3941& 0.7337  &4.5419 
	\end{array} 
	\end{equation*}
	\subsection{(d)}
	
	\subsection{(e)}
	\subsection{(f)}
\end{document}
